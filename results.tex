\documentclass{article}
\usepackage{graphicx} % Required for inserting images

\usepackage{a4wide}
\usepackage{braket}
\usepackage{fancyhdr}
\usepackage[bookmarks]{hyperref}
\usepackage{natbib}
\usepackage{amsmath}
\usepackage{amssymb, subfigure}
\usepackage{amsbsy}
\usepackage{comment}
\usepackage{verbatim}
\usepackage{mathrsfs}
\usepackage{appendix}
\usepackage{bm}
\usepackage{bibentry}
\usepackage[dvipsnames]{xcolor}

\title{Mereology+}
\author{}
\date{June 2024}

\begin{document}

\maketitle

\section{MDS with results from locality from chaos paper}

We start from couplings from the PNAS paper "Unveiling Order from Chaos by approximate 2-localization of random matrices". These couplings are the $J^{\alpha\beta}_{ab}$ that contruct the 2-local Hamiltonian $H'$ with spectrum close to a Hamiltonian $H$ taken from the Gaussian orthogonal enssemble. 
\begin{eqnarray}
    H'=\sum_{ij,\alpha\beta}J^{\alpha\beta}_{ij}\sigma^{\alpha}_i\sigma^{\beta}_j
\end{eqnarray}



\subsection{MSD with couplings}
To do multidimensional scaling using the couplings, we apply the recepei of the paper "Space from Hilbert Space: Recovering Geometry from Bulk Entanglement" skipping a few steps since we are not working with the entanglement structure.

For a simulation the output is 5 tables of couplings $J_{xx}, J_{yy}, J_{zz}, J_{zx}, J_{xz}$, we define the table $J=J_{xx}^2 +J_{yy}^2+J_{zz}^2+J_{zx}^2+J_{xz}^2$ which tells use how strongly 2 qubits interact with each other.

\begin{figure}
    \includegraphics[options]{name}
\end{figure}

This can be seen as a weighted graph where the nodes are qubits and the edges are weighted by the interaction strenght between the qubits.

\begin{figure}
    \includegraphics[options]{name}
\end{figure}

Like in "Space from Hilbert Space" we re-scale the graph by mapping the weights $J_{ab}\rightarrow w_{ab}(J_{ab})=-log(J_{ab}/J_{max})$ with $J_{max}=max(J)$ and $w_{ab}=0$ if $a=b$.
This does have the undesirable effect of making the 2 qubits with maximal interaction strength have distance 0. But the rest are fine.

\begin{figure}
    \includegraphics[options]{name}
\end{figure}

Following the recepei, we use the rescaled graph to calculate a distance graph by defining the distance between 2 qubits as, 
\begin{eqnarray}
    d(a,b)=min_P(\sum_{i} w(p_i,p_{i+1}))
\end{eqnarray}
where $P$ are possible paths between node $a$ and $b$.

\begin{figure}
    
\end{figure}
Note that qubits 1 and 2 are the same. this is because they have the highest interaction strength and their respective distance is artificially put to 0.

From these distance between qubits we would like to use classical multidimensional scaling to find possible coordinates for our qubits in a euclidean manifold of dimension D. 
But from our distance matrix, we get negative eigenvalues which means that the recepei for the coordinate matrix X used in the paper is not useful here.

Using the MDS function from the sklearn package which is based on minimizing the stress when attempting to embed the data in a certain dimension. 

The stress funciton used is ...

We attempt to embedd the qubits into all dimensions from 1 to N and plot the resulting stress.
\begin{figure}
    
\end{figure}
Note we averaged the stress over 100 simulations

SImilarly, for 100 simulations of the N=12 qubits system, we collect all qubits and for an embedding in 3D we plot 

From the distance matrix, we get matrix B,
$$B_{pq}=-\frac{1}{2}(d(p,q)^2-\frac{1}{N}\sum^{N}_{l=1}d(p,l)^2-\frac{1}{N}\sum^{N}_{l=1}d(l,q)^2+\frac{1}{N^2}\sum^N_{l,m=1}d(l,m)^2)$$

From $B$ we contruct $X$ a matrix with columns corresponding to dimensions $x,y,z,...$ and rows the respective coordinate of a qubit in these dimensions.
$$X=(\sqrt{\lambda_1}v_1,\sqrt{\lambda_2}v_2...)$$

note that Classical MDS does not work if B has negative eigenvalues. Which is does.


\subsection{MDS with mutual information}

\subsection*{Other stuff}

Add images
Add what was before on the overleaf
Add discussion points in my TODO.
Make README ok.

potential avenues of exploration. 
Go to larger subsystem sizes. To computational expensive
Define more complex cost functions.



THe rescaling function might be biasing...

\end{document}
